% Copyright 2015 BarD Software s.r.o
% This work is licensed under a Creative Commons Attribution-ShareAlike 4.0
% International License.
% It is loosely based on Presento Beamer theme created by Ratul Saha

% This package defines a few commands which might be useful for 
% building attractive presentations.

\RequirePackage{PresentaxaDeps}
\RequirePackage{textpos}
\RequirePackage{xparse}
% This defines width and height units for text positioning
\setlength{\TPHorizModule}{1cm}
\setlength{\TPVertModule}{1cm}

% Colored square
\newcommand\crule[1][black]{\textcolor{#1}{\rule{2cm}{2cm}}}

%%%%%% \PresentaxaFullPageItemize
%
% when used after \begin{itemize} makes
% itemize filling the whole page height.
\newcommand{\PresentaxaFullPageItemize}{\setlength{\itemsep}{\fill}}

%%%%%% \PresentaxaColors 
%
% defines colors used in Presentaxa theme. Might be useful 
% if you're switching themes with \usetheme
\newcommand{\PresentaxaColors}{\RequirePackage{PresentaxaColors}}

%%%%%% \PresentaxaTextCard[background=colorgreen][foreground=white]{text}
%
% creates a slide filled with background color and places a huge 
% centered text in the middle
\DeclareDocumentCommand{\PresentaxaTextCard}{O{colorgreen} O{white} m}{
  {\setbeamercolor{background canvas}{bg=#1}
    \setbeamertemplate{footline}{}
    \begin{frame}
    \vfill
    \begin{center}
    \hugetext{\ptsansfont \color{#2} #3 \par}
    \end{center}
    \vfill
    \end{frame}
  }
}

%%%%%% \PresentaxaImageCard[opacity=1]{image file}[content={}] ======
%
% creates a slide with the specified image in the background color, applies
% specified opacity and adds content. Everything but image file name is optional.
%
%
\DeclareDocumentCommand{\PresentaxaImageCard}{O{1} m O{}}{
    \setbeamertemplate{footline}{}
    \usebackgroundtemplate{%
    \tikz[overlay,remember picture] \node[opacity=#1, at=(current page.center)] {
      \includegraphics[width=\paperwidth]{#2}};
    }
    \begin{frame}
    #3
    \end{frame}
}

%%%%%% \PresentaxaProgressBar[color=colororange][height=0.2]
%
% creates a progress bar at the frame footline. It grows as presentation 
% progresses
\DeclareDocumentCommand{\PresentaxaProgressBar}{O{colororange} O{0.2}}{
    \newcounter{totalfr}
    \setbeamertemplate{footline}{
        \setcounter{totalfr}{\inserttotalframenumber} 
        \tikz{
          \fill[fill=#1]  (-0.3,0) rectangle (
              {\value{framenumber}*\textwidth/(\value{totalfr})}, #2);
        }
    }
}

