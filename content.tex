% This file demonstrates the usage of Presentaxa Beamer theme
% Copyright 2015 BarD Software s.r.o
% This work is licensed under a Creative Commons Attribution-ShareAlike 4.0
% International License.
\begin{frame}{Тема Presentaxa}
\largetext{
\begin{itemize}
    \PresentaxaFullPageItemize
    \item Приятный внешний вид

        \begin{itemize}
        \setlength{\itemsep}{0.25ex}
        \item ничего лишнего
        \item красивые кириллические шрифты
        \item светлая спокойная цветовая гамма
        \end{itemize}
    \pause
    \item Основана на теме Presento
  
        \begin{itemize}
        \setlength{\itemsep}{0.25ex}
        \item автором оригинальной темы Presento является Ratul Saha. \url{http://www.ratulsaha.com}
        \item переработка и адаптация к кириллице сделаны командой Папирии. \url{http://papeeria.com}
        \end{itemize}
\end{itemize}
}
\end{frame}

\begin{frame}{Отличия от Presento}
\begin{itemize}
    \PresentaxaFullPageItemize
    \item Presentaxa используется стандартным способом: \texttt{\textbackslash usetheme\{presentaxa\}}
    \item Шрифты без кириллических глифов заменены схожими свободными кириллическими
    \item Добавлена поддержка подзаголовков (\texttt{\textbackslash framesubtitle}) и заменен индикатор прогресса
    \item Упрощены структура файлов и набор команд
\end{itemize}
\end{frame}

\begin{frame}{Свободные шрифты}
\begin{itemize}
    \PresentaxaFullPageItemize
    \item {\ptsansfont Название презентации набрано шрифтом PT Sans Caption}
    \item {\latolightfont Заголовки используют шрифт Lato Light}
    \item {\notosansfont Основной шрифт Noto Sans}
    \item {\dejavumonofont Моноширинный шрифт DejaVu Sans Mono}
    \item \textsc{Капительный шрифт Idealist Sans Light}
\end{itemize}
\end{frame}

\begin{frame}{Цветовая палитра}
\begin{center}
    \crule[colordgray] 
    \crule[colorhgray] 
    \crule[colorblue] 
    \crule[colorgreen] 
    \crule[colororange]
\end{center}
\end{frame}

\PresentaxaTextCard{ЭТО ОЧЕНЬ ВАЖНЫЙ СЛАЙД!}

\PresentaxaImageCard[0.7]{images/skeleton}[
    \begin{textblock}{7}(5,-3)
    \color{colorblue}\hugetext{\textbf{Погнали!}}
    \end{textblock}
]